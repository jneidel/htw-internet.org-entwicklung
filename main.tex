\documentclass{article}
\usepackage[backend=biber]{biblatex}
\addbibresource{bibliography.bib}

\title{Wie entwickelte sich Facebooks Internet.org/Free Basics in Indien?}
\author{
  Fischer, Sarah Susanne\\
  \and
  Hein, Oliver\\
  \and
  Neidel, Jonathan\\
  \and
  Petersen, Tom Magnus\\
  \and
  Schmitz, Kai-Ibe\\
}

\begin{document}

\maketitle
\subsubsection {Facebook}
Eine Studie von Neha Kumar, welche zwischen 2011 und 2012 in New Delhi durchgeführt wurde, zeigt, dass Facebook zu der Zeit Facebook eine große Rolle in der Gesellschaftsentwicklung spielte. 

Da zu 2012 eine Regulierung von 200 SMS pro Tag veröffentlicht wurde, stieg das Interesse von Facebook zu der Zeit stark an,
um mit seinen Freunden und Familie weiterhin grenzenlos kommunizieren zu können. 
 
Von den 135 Millionen indischen Facebook Nutzern waren 2015 gerademal 25\% weiblich, da vorallem jungen indischen Frauen, aus elterlichen Prinzipien, 
wenig Freiheit gelassen wurde, sich mit Freunden zu treffen, oder gar Technologie
zu besitzen. Dieser Anteil sollte sich laut Aktivisten in der Zukunft stark ändern, aber zu 2018 sind sogar 78% der indischen Facebook Nutzer männlich. 
Das rührt aber auch daher, dass auf Facebook keiner den Bildungsstand hinterfragt, den man angibt, was zu der Normalität führte, seinen Lebenslauf besser darzustellen,
als er eigentlich ist. Es sei nicht selten, sich mehr als einen Account zu machen und auf den Zweitaccounts Profilbilder
von Bollywood-Stars zu benutzen, mit welchem sie sich erhöhte Aufmerksamkeit erhoffen.
Anfangs machten Freunde unter sich sogar Wettbewerbe aus, wer mehr Freunde in kurzer Zeit hinzufügen könne.
Diese Trends ließen aber nach längerer Facebooknutzung nach und viele brachten sich Englisch, die "Sprache des Internets", 
selbst in ihrer Freizeit bei, um besser mit internationalen Kontakten
kommunizieren zu können und auch ausländische News und Medien zu verstehen, ohne Google Übersetzer benutzen zu müssen. 


% \printbibliography
\end{document}
