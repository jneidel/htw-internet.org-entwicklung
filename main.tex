\documentclass{article}
\usepackage[backend=biber]{biblatex}
\usepackage{packages/timeline}
\addbibresource{bibliography.bib}

\title{Wie entwickelte sich Facebooks Internet.org/Free Basics in Indien?}
\author{
  Fischer, Sarah Susanne\\
  \and
  Hein, Oliver\\
  \and
  Neidel, Jonathan\\
  \and
  Petersen, Tom Magnus\\
  \and
  Schmitz, Kai-Ibe\\
}

\begin{document}

\maketitle

% 2 in der Gliederung
\section{Lebenszyklus der Initative}

Nachdem nun die Ausgangslage geschildert wurde, wird in diesem Teil der Verlauf des Projektes von der Einführung bis zum Verbot nach verfolgt.```

\subsection{Übersicht}

\definecolor{BGColor}{rgb}{1.0,1.0,1.0}
\begin{timeline}{2015}{2017}{250}{145}
  \MonthAndYearEvent{2}{2015}{Formeller Start}
  \MonthAndYearEvent{4}{2015}{Partielle Einführung}
  \MonthAndYearEvent{7}{2015}{Umbenennung zu Free Basics}
  \MonthAndYearEvent{11}{2015}{Vollständige Einführung}
  \MonthAndYearEvent{2}{2016}{Verbot}
\end{timeline}

Nur ein Jahr hat es vom offizellen Start Anfang 2015, bis zum Verbot im Februar 2016 gedauert.

In der Zwischenzeit versuchte Facebook, nach initellen Startschwierigkeiten, die Initiative durch Rebranding zu retten, was aber nicht reichte um die kritische Öffentlichkeit zu beeinflußen, deren Meinung bis zu den Entscheidungträgern der zuständigen Behörde vordrigen konnte.
Welche sich dann mit einem indirekten Verbot auf der Seite der Protestierenden positionierte.

% \printbibliography
\end{document}
