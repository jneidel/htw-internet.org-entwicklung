\documentclass{article}
\usepackage[backend=biber]{biblatex}
\addbibresource{bibliography.bib}

\title{Wie entwickelte sich Facebooks Internet.org/Free Basics in Indien?}
\author{
  Fischer, Sarah Susanne\\
  \and
  Hein, Oliver\\
  \and
  Neidel, Jonathan\\
  \and
  Petersen, Tom Magnus\\
  \and
  Schmitz, Kai-Ibe\\
}

\begin{document}

\maketitle
\subsubsection {Facebook}
Eine Studie von \textcite{empowermentThroughFacebook}, welche zwischen 2011 und 2012 in New Delhi durchgeführt wurde, zeigt, dass Facebook zu der Zeit eine große Rolle in der Gesellschaftsentwicklung spielte. 
\medskip
Da zu 2012 eine Regulierung von 200 SMS pro Tag veröffentlicht wurde, stieg das Interesse von Facebook zu der Zeit stark an,
da Handybesitzer weiterhin unlimitiert mit ihren Freunden kommunizieren wollten.
Durch das Kennenlernen und Nutzen von Facebook bildete sich eine Gesellschaft, welche bislang der Globalisierung
durch das Internet ausgeschlossen waren und nun erstmals eine transnationale Identität erstellen konnte.
\medskip
Bis 2015 wuchs die Nutzeranzahl von Facebook in Indien geradezu explosionsartig, und mit derzeit 135 Millionen Nutzer waren gerademal 25\% weiblich. Das liegt vorallem daran, dass vorallem jungen indischen Frauen, aus elterlichen Prinzipien, 
wenig Freiheit gelassen wurde sich mit Freunden zu treffen, oder gar Technologie
zu besitzen. Dieser Anteil sollte sich laut Aktivisten in der Zukunft stark ändern, 
aber zu 2018 sind sogar 78\% der indischen Facebook Nutzer männlich. 
\medskip
Das rührt aber auch daher, dass auf Facebook keiner der Nutzer den Bildungsstand hinterfragt, den man angibt, was zu der Normalität führte, seinen Lebenslauf besser darzustellen, als er eigentlich ist. Es sei nicht selten, sich mehr als einen Account zu machen und auf den Zweitaccounts Profilbilder von Bollywood-Stars, oder anderen Berühmtheiten zu benutzen,
mit welchem sie sich erhöhte Aufmerksamkeit erhofften. 
Anfangs machten Freunde unter sich sogar Wettbewerbe aus, wer mehr Freunde in kurzer Zeit hinzufügen könne.
Diese Trends ließen aber nach längerer Facebooknutzung nach und viele brachten sich Englisch, die "Sprache des Internets", 
selbst in ihrer Freizeit bei, um besser mit internationalen Kontakten
kommunizieren zu können und auch ausländische News und Medien zu verstehen, ohne Google Übersetzer benutzen zu müssen.
\medskip
Dennoch blieb die Internetgeschwindigkeit ein Problem, mit welcher die Inder eine durchschnittliche Seitenladezeit von ganzen 3,9 Sekunden hatten\autocite{mashable}. Anfang 2015 veröffentlichte Facebook die Facebook Messenger App, mit welcher sich nur Nachrichten und Bilder schicken ließen, und die Internetverbindung demnach nicht allzusehr belastet wird, aber eine Lösung schien das nicht zu sein.
\medskip
%Mit dieser riesigen Facebook Nutzerzahl und der stark unterdurchschnittlichen Internetgeschwindigkeit war Indien das perfekte Ziel für



% \printbibliography
\end{document}
