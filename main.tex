\documentclass{article}
\usepackage[backend=biber]{biblatex}
\addbibresource{bibliography.bib}

\title{Wie entwickelte sich Facebooks Internet.org/Free Basics in Indien?}
\author{
  Fischer, Sarah Susanne\\
  \and
  Hein, Oliver\\
  \and
  Neidel, Jonathan\\
  \and
  Petersen, Tom Magnus\\
  \and
  Schmitz, Kai-Ibe\\
}

\begin{document}

\maketitle

% 1.2.4 in der Gliederung
\subsubsection{Aktivisten}

% go into their arguments?

Die Seite welche gegen Facebook argumentiert sind die 'Net Neutrality' Aktivisten, vereint under dem Banner der 'Save the Internet' Kampagne.
Zusammengesetzt aus Technologieinteressierten, primär den Mitarbeitern von Tech-Startups, die sich selbst als ``geeks and enthusiasts from various fields: technology, law, journalism, design, policy" \cite{sti2015} beschreiben.

% recursive public

% their position in society

In Ihrer öffentlicher Präsentation ist die Initiative angelehnt an die 'Net Neutrality' Bewegung der USA: von der Namensgebung und dem Logo über die verwendeten Hashtags in den sozialen Medien bis zu dem Vokabular mit welchem sie arbeiten (Vgl. \cite{prasad2017}). Diese bewusste gewählte Assoziation mit dem erfolgreichen amerikanischen Modell und auch die direkten Verweise auf westliche Popkultur soll die Kampagne für Ihre Zielgruppe attraktiver machen.

Die Zielgruppe der 'Save the Internet' Kampagne sind nicht die Menschen denen Free Basics helfen möchte, sondern die indische Mittelklasse. Denn die Sprache mit der die Menschen angesprochen werden ist 'Hinglish', ein Mix aus Hindi und Englisch, für deren Verständnis sowohl Englischkentnisse - Flüssigkeit in Englisch, welche Zugang zu neuen, qualifizierten Jobs ermöglicht, und eine Nähe zum Westen definieren die neue Mittelklasse Indiens (Vgl. \cite{fernandes2006}) - als auch ein hohen Sprachniveau in Hindi vorraussetzt.

Ironischerweise, obwohl die Aktivisten gegen Facebook wettern, verbreiten sie Ihre Botschaft fast ausschließlich über soziale Medien, also auch zu großen Teilen über Facebook. Andere Kanäle um auf Ihre technopolitische (politische Meinungsbildung, getrieben durch Technologie) Kampagne aufmerksam zu machen sind u.a. (Vgl. \cite{prasad2017}):
\begin{itemize}
  \item YouTube: z.B. durch Erklärvideos einer angeheuerten komödischen Gruppe
  \item Ihr eigener Blog: in welchem die Probleme in einer seriöseren Form geschildert werden
  \item Twitter:
    \begin{itemize}
      \item zur direkten Interaktion untereinander und mit den Anhängern
      \item der Konfrontation mit Befürwörtern von Internet.org
      \item um bei Tech-Firmen Gutheißungen für die Bewegung zu erbitten
    \end{itemize}
\end{itemize}

% \printbibliography

\end{document}
