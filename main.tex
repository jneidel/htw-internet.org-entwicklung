\documentclass{article}
\usepackage[backend=biber]{biblatex}
\addbibresource{bibliography.bib}

\title{Wie entwickelte sich Facebooks Internet.org/Free Basics in Indien?}
\author{
  Fischer, Sarah Susanne\\
  \and
  Hein, Oliver\\
  \and
  Neidel, Jonathan\\
  \and
  Petersen, Tom Magnus\\
  \and
  Schmitz, Kai-Ibe\\
}

\begin{document}

\maketitle
% 2.3 in der Gliederung
\subsection{Start in ausgewählten Regionen}

Im Anschluss an den formalen Start fingen erneute Debatten um "Net-Neutrality" an Beoden zu gewinnen. Facebooks Dienst sei nicht konform mit dem Bestreben eines freien Internets, trotz der Bewerbung des Produkts als "kostenloses Facebook"\cite[3]{prasad2017}
 , später als ein "step towards digital equality".

\medskip

In der ländlichen Bevölkerung gab es auch geteilte Meinungen hinsichtlich des Zugangs zum Internet, zumindest bei jenen, die sich selbst informiert haben oder (von zum Beispiel Nachrichtendiensten wie "Digit.in") \textcite{digitYT}
informiert werden. Hier kommt es jdeoch zumeist zur Ignoranz des großen Ganzen, der Dienst wird als positiv entgegengenommen und genutzt, Aktionen wie die wortwörtlich vorgeschriebene Unterschreibung zugunsten von Facebook in der Debatte um Netzneutralität \textcite{ndtvYT}
werden ohne weiteres Hinterfragen und potenziell aufgrund von Unwissen unterstützt. Einzig ein initielles Interesse an einem kostenlosen Dienst im allgemeinen Sinne ist gemein.

\medskip

Anzumerken bei alledem ist, dass der Großteil der Bevölkerungsgruppe, die in Facebooks Werbekampanien als Zielgruppe des Dienstes beschrieben werden, entweder keinen oder nur einen gerinfügigen Nutzen genießt. Jene, die sich den Besitz eines digitalen Endgeräts (id est Smartphones, etc.) leisten können, sind aufgrund der geringen Preise auch in der Lage, eine stabilere und schnellere Internetverbindung als Facebooks Dienst sie anbietet zu erwerben. Der Rest, also Personen, die (zu Teilen weit) unterhalb der Armutsgrenze leben, haben weder Wege, FreeBasics zu nutzen, noch einen Grund dafür. \cite[257]{everydayLife}

\medskip

So ist die Nutzung des Dienstes selbst fern vom beinahe utopischen Beispiel von Facebooks Seite des Landwirts Ganesh, welcher mithilfe von Free Basics seinen Ertrag verdoppelte, indem er Informationen über das Wetter und aktuelle Warenpreise nutzte, ein Beispiel für die Angabe, dass für alle 10 Teilnehmer an Internet.org eine Person aus der Armut gehoben wird. \cite[4]{prasad2017}
In der Realität besteht die Nutzerbasis wie besagt aus männlichen Jugendlichen und jungen Erwachsenen. Hier wurde Facebook zum Teil ein integrierter Teil des Lebens, das sich fast virusartig durch persönliche Weiterempfehlung verbreitete. Dies geht über zu spezialisierten Kursen von neuen oder schon vorher "Tech-Affinen", die Teilnehmern Grundlegendes von der Inbetriebnahme von Smartphones bis zum Kommentieren bei Facebook beibringen. \cite{empowermentThroughFacebook}

\medskip

Diese Verbreitung zog selbstverständlich eine Verschiebung der sozialen Normen nach sich, besonders bei Fragen der sozialen Akzeptanz unter den Jugendlichen. Der Besitz eines "guten", sprich aktuellen Smartphones und viele (auch internationale) Freunde auf Facebook wurden Teil des neuen Selbst, ähnlich wie es schon im Westen der Fall war. Es folgte eine Mentalität des ewigen Kreislaufs des Neids; "I want what he has". \cite{empowermentThroughFacebook}

\medskip

 Ein weiterer Aspekt ist die Misrepräsentation der eigenen Person 'online', was immer mehr zur Norm wurde. So wird ein Uniabbruch plötzlich zum bestandenen Bachelor, ein solcher zum sogennanten "Honours degree", eine "höher gestellte Variante". Ursprung dieser Entwicklung sei ein allgemeines Gefühl der Nichtdazugehörigkeit, die Generation der gelehrten, arbeitslosen zwanzig- bis dreißigjährigen Indern, eine Generation, die von der Welt "abgehängt" wurde. Das Lehren des Internets an Andere wird hier zum Daseinszweck, gleiches gilt für die Verbindung zu internationalen Freunden via 'Facebook' und 'Google Translate'. \cite{empowermentThroughFacebook}

\medskip

Besagte Kurse brachten einen weiteren positiven Effekt mit sich, die Verbreitung des Englischen durch Selbstbeibringung in Gebieten, in denen vorher weder Englisch noch Hindi gesprochen wurde. \cite{empowermentThroughFacebook} Die Tatsache, dass der Großteil der Analphabeten vor Allem unter der genannten Gruppe, welche unter der Armutsgrenze lebt, zu finden sind, bleibt hierbei leider bestehen.

\medskip

Ferner stehen positiven Aspekten weitere bereits etablierte soziale Normen im Wege; Informationsverbreitung geschieht im ländlichen Indien immer noch zu großen Teilen über Mundpropaganda \cite[259]{everydayLife}, Mädchen wird Zugriff auf Medien zugunsten des (erstgeborenen) Jungen verwährt \cite{empowermentThroughFacebook}, soziale Gruppierungen werden vom Realen ins Virtuelle übernommen.

\medskip

All dies hätte sich nicht durch eine kostenpflichtige Variante des Internets geändert. Allein die Zahlen der nationalen Internetverbreitungsrate (2016 25\% \--> 2019 50\%) sprechen für eine graduelle Veränderung gen Akzeptanz bezüglich des Internets. %I think this should be left out - or changed significantly - need opinions

\parencite[Vgl.]{prasad2017}
\parencite{digitYT}
\parencite{ndtvYT}
% \printbibliography
\end{document}
