\documentclass{article}
\usepackage[backend=biber]{biblatex}
\addbibresource{bibliography.bib}

\title{Wie entwickelte sich Facebooks Internet.org/Free Basics in Indien?}
\author{
  Fischer, Sarah Susanne\\
  \and
  Hein, Oliver\\
  \and
  Neidel, Jonathan\\
  \and
  Petersen, Tom Magnus\\
  \and
  Schmitz, Kai-Ibe\\
}

\begin{document}

\maketitle
% 1.2.2 in der Gliederung
\subsubsection{Internetverbreitung}

In Indien leben 2015 schon 350 Millionen Menschen\autocite{slideshareIndia}, welche das Internet aktiv nutzen, und davon sind 159 Millionen per Handy im Internet unterwegs. Somit sind sie schon auf Platz 2 der aktiven Internetnutzer weltweit. So groß sich die Zahl auch anhören mag, das sind lediglich 27\% der Bevölkerung\autocite{InternetCountry}. 

Ein Computer ist in Indien ein Luxusgut, welches durch die rasante Entwicklung von Mobilgeräten für die Jugend schnell an Bedeutung verlor, während Handys mit mobilem Internet allgegenwärtig wurden.
Durch die explosionsartige Verbreitung der neuen Kommunikationsmöglichkeiten entstanden in kürester Zeit große Gruppen im Internet, welche sich über neue Technologien, Wissenswertes und Ihre Interessen auszutauschen.
Dieser rapide Aufbau von Gesellschaften ist ein neues Phänomen, welches vorallem in `Entwicklungsländern' auftritt, in denen neue, und vergleichsweise preiswerte Technologien, den Zugang zum Internet ermöglichen.   \textcite{empowermentThroughFacebook}

\medskip
Die durchschnittliche Bandbreite mobilen Internets war mit 2.8 Mb/s der durchschnittlichen Bandbreite eines Festanschlusses, von 2.3 Mb/s, überlegen \autocite{slideshareIndia}.
Da der Preis einer Prepaid Karte, mit 1 GB Datenvolumen, ca. 3.55\% des durchschnittlichen Monatsgehalt beträgt \autocite{broadbandAgency}, können sich hauptsächlich nur die Inder Internet leisten, welche in Stadtregionen leben, da dort das Einkommen weitaus höher ist.

 Der Großteil der indischen Bevölkerung lebt aber im ländlichen Gebiet  und gerademal 31,2 \% \autocite{IndiaBevölkerung} in städtischen Gebieten. 
Von den 350 millionen aktiven Internetnutzern leben also lediglich 61 millionen nicht in einer Stadt. Somit benutzten 2015 gerademal 7.5 \%, auf dem Land lebende Inder, aktiv das Internet \autocite{slideshareIndia}.
Grund dafür ist vorallem die fehlende Abdeckung mobilen Internets in ländlichen Gebieten und der hohe Preis, was dazu führt, dass vorallem Jugendliche vom Land in städtische Gebiete ziehen, um dort bessere Bildung und bessere Arbeitsplätze zu beziehen und somit die ländlichen Gebiete noch mehr dem Armut verfallen. 

%Überleitung zu Internet.org - > rural poor people + Internet.org = rural rich people?

% \printbibliography
\end{document}
