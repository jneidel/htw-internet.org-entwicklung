\documentclass{article}
\usepackage[backend=biber]{biblatex}
\addbibresource{bibliography.bib}

\title{Wie entwickelte sich Facebooks Internet.org/Free Basics in Indien?}
\author{
  Fischer, Sarah Susanne\\
  \and
  Hein, Oliver\\
  \and
  Neidel, Jonathan\\
  \and
  Petersen, Tom Magnus\\
  \and
  Schmitz, Kai-Ibe\\
}

\begin{document}

\maketitle

% 1.2.1
\subsubsection{Facebook}

Facebook erlebte ab 2011 ein starkes Wachstum an Nutzern in Indien, was 2012 mitunter durch die Beschränkung des Versendens von SMS auf maximal 200 pro Tag durch die TRAI unterstützt wurde, da Handybesitzer weiterhin unlimitiert kommunizieren wollten.
Dieses Kennenlernen und Nutzen von Facebook brachte den Teil der indische Gesellschaft, welcher bislang von der Globalisierung des Internets ausgeschlossen war, erstmals zur Bildung einer gemeinsamen, transnationalen Identität, was im Kontrast zur immer noch währenden Kastengesellschaft stand und steht.
Jene verwähren besonders Frauen den Zugriff auf soziale Medien und das Internet allgemein, die hohe Anzahl an Männern unter den Nutzern sei eine Gefahr für deren Keuschheit, doch selbst bei verheirateten Frauen wird häufig der Internetgebrauch durch zum Beispiel die Schwiegereltern überwacht.

\medskip

Besagte Überrepräsentation von jungen Männern zog unter anderem die Verbreitung der Misrepräsentation der eigenen Person online nach sich, was immer mehr zur Norm wurde.
Dies beinhaltet zumeist den Bildungsstand, den sozialen Hintergrund und Profilbilder. So wird ein Uniabbruch plötzlich zum bestandenen Bachelor, ein solcher zum sogennanten `Honours degree', eine höher gestellte Variante.

\medskip

Das Internet brachte jedoch auch Positives mit sich: da selbst gebildete Inder kaum mit der englischen Sprache vertraut waren, wurde das Erlernen und Lehren Anderer von der `Sprache des Internets" allgegenwärtig.
So bildeten sich Lerngruppen, welche sich in Internetcafe's trafen, um dort nicht nur Wissen, sondern auch den vergleichsweise teuren Datentarif zu teilen.

Neben diesen bleib auch die Internetgeschwindigkeit ein Problem, mit welcher die Inder eine durchschnittliche Seitenladezeit von 3,9 Sekunden hatten\autocite{mashable}.
Facebook versuchte auf dieses Problem mit der Messenger App einzugehen, in welcher nur Chats geladen werden, was aber nicht die endgültige Lösung darstellte.
Als eine weiterer Lösungsansatz gilt das Konzept für Internet.org.

% \printbibliography
\end{document}
