\documentclass{article}
\usepackage[backend=biber]{biblatex}
\addbibresource{bibliography.bib}

\title{Wie entwickelte sich Facebooks Internet.org/Free Basics in Indien?}
\author{
  Fischer, Sarah Susanne\\
  \and
  Hein, Oliver\\
  \and
  Neidel, Jonathan\\
  \and
  Petersen, Tom Magnus\\
  \and
  Schmitz, Kai-Ibe\\
}

\begin{document}
\section{Facebooks Absicht/Gründe für Internet.org}
Facebook, ein soziales Netzwerk, wurde 2004 auf den Markt gebracht und seit da an ist es, und das dahinterstehende Unternehmen Facebook, Inc., stetig gewachsen. 
Doch da die Anzahl von Anmeldungen neuer Internet-Kabelanschlüsse in der westlichen Welt seit 2008 gesunken sind,
(Vgl. \cite{ICTslowingDown}) hat das Wachstum von Facebook dementsprechent auch nachgelassen.\\
Dies passiert obwohl ``at the beginning of 2016, only an estimated 3.2 billion people — 44 percent of the world’s population are online and connected to the digital economy." (\cite{connectWorld})\\
Der Grund dafür ist, dass der Großteil der restlichen 56\% aus Menschen besteht, die in den Entwicklungsländern leben. Diese haben entweder nicht die Möglichkeit das Internet zu benutzen (wegen einen zu teuren/keinen Internetzugang) oder sehen nicht den Nutzen des Internets, da er ihnen nie vermittelt worden ist.

\medskip

Daraus lässt sich ableiten, dass für den weiteren Wachstum Facebooks es nötig ist, dass das Internet in den Entwicklungsländern ausgebreitet, und die Interesse der Menschen dort für die Anwendung geweckt wird

 @jneidel
jneidel 3 hours ago Owner
Vlt. Bezug nehmen auf den Anspruch des ewigen Wirtschaftswachstums (trotz limitierter Resourcen).
\medskip
Mark Zuckerberg, Gründer und Vorstandsvorsitzender von Facebook Inc., hat 2013 das Dokument "Is Connectivity a Human Right?" veröffentlicht.
In diesem, welches die Grundlage für Internet.orgs Konzept ist, erläutert er, dass es für Facebook wahrscheinlich nicht gewinnbringend sei, das Internet in den Entwicklungsländern auszubreiten. Trotzdem arbeitet Facebook daran dies zu machen, da er daran glaubt, dass jeder Mensch es verdient hat verknüpft zu sein.(Vgl. \cite{HumanRight})

\medskip

Positive Auswirkungen durch eine Ausbreitung des Internets, welches eine Bildungs- und Verbindungsmöglichkeit darstellt, in Ländern der 3. Welt, sind die Entstehung neuer Arbeitsplätze, der Rückgang der Armut und die positive Entwicklung der Wirtschaft.\\
    
Diese Auswirkungen werden in einem Report, welcher von Facebook zur Unterstützung des Internet.org Projektes in Auftrag gegeben wurden ist, global verdeutlicht:
                    
``[...] we believe that achieving universal Internet penetration could expand world output by \$6.7 trillion.  
In addition, global inclusion could bring 7 percent of the world’s population above absolute poverty levels — in effect providing 500 
million of the world’s poorest people new opportunities to grow and engage with the world"
\parencite{connectWorld}

\medskip

Mark Zuckerberg ist der Meinung, dass die Grundlage für eine zukünftige wissenbassierte Wirtschaft, die globale Vernetzung der Menschen ist.
(\cite{HumanRight})
Die Menschen, die man online bringen würde, würden viele neue Ideen, Dienstleistungen und Produkte bereitstellen, durch welche die ganze Welt profitieren würde.
% \printbibliography
\end{document}
