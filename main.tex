\documentclass{article}
\usepackage[
backend=biber,
style=authoryear,
citestyle=authoryear,
autocite=footnote
]{biblatex}
\usepackage{packages/timeline}
\addbibresource{bibliography.bib}

\title{Wie entwickelte sich Facebooks Internet.org/Free Basics in Indien?}
\author{
  Fischer, Sarah Susanne\\
  \and
  Hein, Oliver\\
  \and
  Neidel, Jonathan\\
  \and
  Petersen, Tom Magnus\\
  \and
  Schmitz, Kai-Ibe\\
}

\begin{document}

\maketitle

% 1
\section{Ausgangsbedingungen}

Zunächst wird ein Überblick über die Ausgangsbedingungen in Indien geschaffen. Von der Internetverbreitung, Ihrer Beziehung zu Facebook, vorherigen Diskussionen zur Netzneutralität sowie einer Beschreibung der Aktivisten, welche sich gegen Facebooks Initiative wendeten.

Als Erstes wird nun aber das Konzept für Internet.org/Free Basics vorgestellt, gefolgt von einer Beleuchtung von Facebooks Beweggründe für die Gründung der Initiative.

% 1.1
\subsection{Internet.org/Free Basics}

% 1.1.2
\subsubsection{Konzept}
        
Mark Zuckerberg nennt drei Punkte, an denen man arbeiten muss, damit sich das Internet in den Entwicklungsländern ausbreiten kann
\parencite{HumanRight}:

\begin{itemize}
\item ``making internet access affordable by making it more efficient to deliver data."
\item ``using less data by improving the efficiency of the apps and experiences we use."
\item ``helping businesses drive internet access by developing a new model to get people online."  
\end{itemize}

Die ersten zwei Punkte, die er erwähnt, zielen darauf ab das Internet kostengünstiger/kostenlos anzubieten. 
Dies ist notwendig damit Menschen, die nur ein geringes Einkommen haben, es auch nutzen können.

Dazu ist es zunächst nötig den Datenverkehr effizienter gestalten, zum Beispiel durch leistungsfähigere Hardware.
Dies senkt Infrastrukturkosten für die Unternehmen, die diese bereitstellen.
Wenn dieser Profit an die Kunden weitergereicht wird führt das zu einer Senkung der Internetkosten für den Einzelnen.

Zusätzlich könnte das dadurch entstehende überschüssige Kaptial in den weiteren Ausbau des Netzwerkes investiert werden.

\medskip

Im zweite Punkt konzentriert sich Zuckerberg auf die Erhöhung der Leistungsfähigkeit von existierenden Dienstleistungen.
Es ist logisch: Weniger übertragene Daten entsprechen geringeren Kosten für den Nutzer.

\begin{quote}
``At Facebook, we’re investing heavily in opportunities to reduce our overall data use and help  other apps     
reduce their data use as well. Some of the areas we’re focused on are caching, data compression and simple efficiency optimizations.
- Mark Zuckerberg \textcite[8]{HumanRight}
\end{quote}

% cut
Die angesprochene Komprimierung kann man entweder durch die direkte Kompression von großen Apps realisieren oder indem man neue Dienstleistungen entwickelt, durch welche man seine Daten schicken und komprimieren lassen kann.

Facebook hat diese Mittel verwendet um den Datenverkehr effizienter zu gestalten.
Die Projekte Facebook Zero (Bereitstellung einer kostenlosen Internetverbindung, solange man sich auf Facebooks Webseiten befindet) und ``Facebook für jedes Handy" (eine Version von Facebook für feature Phones) wurden jeweils  2010 und 2011 eingeführt.
Facebook will die verwendete Technologie nun allgemein anwendbar machen, damit andere Dienstleistungen diese auch benutzen können.

\medskip

Mit dem letzten Punkt bekräftigt Zuckerberg den Glauben, die Ausbreitung des Internets sei die Aufgabe der Industrie , der Facebook mit Konzepten wie Internet.org zu Seite stehen möchte.

Dieses neue Konzept wurde dermaßen umgesetzt: Menschen in Entwicklungsländern sollen zuerst nur den kostenlosen Zugang zu grundlegenden Dienstleistungen wie Facebook haben.
Für diese Herangehensweise gab Facebook zwei Begründungen:
\medskip
Nicht jede Dienstleistung kann kostenlos bereitgestellt werden, da die dahinterstehenden Anbieter Profit erzielen müssen. 
Aber durch das kostenlose Bereitstellen grundlegender Anwendungen könnte man gewährleisten, dass die meisten Menschen Zugang zum Internet haben und die Industrie trotzdem gleichzeitig den größten wirtschaftlichen Profit erzielt (wodurch diese wiederum mehr Zugang zum Internet bieten kann).
Obwohl Zuckerberg sagt, dass er keine spezifische Gruppe von grundlegenden Internetdienstleistungen vorgibt, beschreibt er, welche Anwendungen er als grundlegend ansieht.

\begin{quote}
``many people who have never experienced the internet don’t know what a data plan is or why they’d want one. 
However, most people have heard of services like Facebook and messaging and they want access to them. If we can provide people 
with access to these services, then they’ll discover other content they want and begin to use 
and understand the broader internet." - \textcite[4]{HumanRight}
\end{quote}

Seiner Meinung nach sind es die Dienstleistungen, die nicht viel Daten verbrauchen (hauptsächlich Text basierende und einfach gehaltene Apps) und durch welche anderer Inhalt entdeckt werden kann.
Dienstleistungen, die unter diese Kriterien fallen, sind zum Beispiel Instant-Messaging-Dienste und soziale Netzwerke, Suchmaschinen und Enzyklopädien wie Wikipedia.(\cite[5]{HumanRight})
Also ist Facebook nach Zuckerbergs Meinung eine grundlegende Internet Dienstleistung, die vorrangig zu anderen Anwendungen kostenlos zur Verfügung gestellt werden sollte.

Aber durch das kostenlose Bereitstellen grundlegender Anwendungen könnte man gewährleisten, dass die meisten Menschen Zugang zum Internet haben und die Industrie trotzdem gleichzeitig den größten wirtschaftlichen Profit erzielt (wodurch diese wiederum mehr Menschen den Zugang zum Internet ermögliche können).
Obwohl Zuckerberg sagt, dass er keine spezifische Gruppe von grundlegenden Internet Dienstleistungen vorgibt, beschreibt er welche Anwendungen er als grundlegend ansieht.

\begin{quote}
``many people who have never experienced the internet don’t know what a data plan is or why they’d want one. 
However, most people have heard of services like Facebook and messaging and they want access to them. If we can provide people 
with access to these services, then they’ll discover other content they want and begin to use 
and understand the broader internet." - \textcite[4]{HumanRight}
\end{quote}

Facebooks Plan ist es diese Punkte unter dem Projekt Internet.org Industrie übergreifend zu bearbeiten.

% 1.2.3
\subsubsection{Netzneutralität}

Netzneutralität, oder `Net Neutrality' im Englischen, beschreibt ein Prinzip, nach welchem ein Internetanbieter allen Datenverkehr gleich behandelt, dass heißt ohne bestimmte Internetangebote langsamer oder teurer zu machen als andere \autocite{netzneutralität}.

\medskip

Indiens Vergangenheit ist im Bezug auf Netzneutralität noch sehr jung. So gab es vor Internet.org keine bestehende Verordnungen, und nur einige Vorfälle, welche eine Diskussion des Themas veranlassten.
Beide relevanten Debatten, ausgelöst vom lokalen Mobilfunkprovider Airtel, zielten darauf entgangene Gewinne des Konzerns ausgleichen zu lassen.
So wurde 2012 die Möglichkeit zur Besteuerung von Youtube, Google, Facebook, und Co. vorgeschlagen um für deren großen Teil am Datenverkehr zu kompensieren.
Auch 2014 ging es um eine ähnliche Problematik: so sollten VoIP (Voice over IP, Internettelefonie) Daten teurer sein als sonstige mobile Daten.
Beide Vorschläge führten zu keinem Ergebnis und wurden zum Teil stark von der Öffentlichkeit protestiert (Vgl. \textcite[253]{everydayLife} und \textcite{airtelVoip}).

% 1.2.4
\subsubsection{Aktivisten}

Die Seite, welche gegen Facebook argumentiert, sind die `Net Neutrality' Aktivisten, vereint unter dem Banner der `Save the Internet' Kampagne.
Zusammengesetzt aus Technologieinteressierten, primär den Mitarbeitern von Tech-Startups, die sich selbst als ``geeks and enthusiasts from various fields: technology, law, journalism, design, policy" \parencite{sti2015} beschreiben.

\medskip

Die Aktivisten können als eine `recursive public' verstanden werden. Welche von \textcite{twoBits} wie folgt definiert wird:

\medskip

``A recursive public is a public that is constituted by a shared concern for maintaining the means of association through which they come together as a public.
Geeks find affinity with one another because they share an abiding moral imagination of the technical infrastructure, the Internet, that has allowed them to develop and maintain this affinity in the first place." - \cite[28]{twoBits}.

\medskip

Dies deckt sich mit dem Handeln der Aktivisten, welche sich über online Plattformen kennengelernt und unter dem gemeinsamen Ziel - das Internet wie sie es kennen (und lieben) zu verteidigen - vereinigt haben.

\medskip

Obwohl die Mitglieder von `Save the Internet' nur einen kleinen Teil der Bevölkerung repräsentieren, nämlich den der high-tech Arbeiter deren Anteil bei 2\% liegt (in einem Land in welchem 70\% der Menschen landwirtschaftlichen Tätigkeiten nachgehen), haben sie doch eine besondere Position inne. Denn sie spielen eine zentrale Rolle in der Vorstellung des neuen Indiens, welches in der globales Wirtschaft mitmischen kann (Vgl. \cite{thomas2012}).
Sie sind das Symbol der aufsteigenden Mittelschicht, welche durch Unternehmerschaft und harte Arbeit die indische Wirtschaft (7,7\% des 2,6 Billionen BIP 2017 \autocite{statistaIndiaGDP}\autocite{imfIndiaGDP}) und Indiens Position in der Welt voran treiben (8\% der Startups im Silicon Valley der 90er Jahre wurden von Indern angeführt \parencite{upadhya2004}).

\medskip

In Ihrer öffentlichen Präsentation ist die Initiative angelehnt an die `Net Neutrality' Bewegung der USA: von der Namensgebung und dem Logo über die verwendeten Hashtags in den sozialen Medien bis zu dem Vokabular mit welchem sie arbeiten (Vgl. \cite{prasad2017}).
Diese bewusst gewählten Assoziation mit dem erfolgreichen amerikanischen Modell, und auch die direkten Verweise auf westliche Popkultur, sollen die Kampagne für Ihre Zielgruppe ansprechender machen.

Die Zielgruppe setzt sich nicht aus den Menschen zusammen welchen Free Basics helfen möchte, sondern aus dem indische Mittelstand. Denn die Sprache mit der die Menschen angesprochen werden ist `Hinglish', ein Mix aus Hindi und Englisch, für deren Verständnis sowohl Englischkenntnisse - welche den Zugang zu neuen, qualifizierten Jobs ermöglichen und, ebenso wie eine Nähe zum Westen, die neue Mittelschicht Indiens prägen (Vgl. \cite{fernandes2006}) - als auch ein hohen Sprachniveau in Hindi voraussetzt.

\medskip

Ironischerweise, obwohl die Aktivisten gegen Facebook wettern, verbreiten sie Ihre Botschaft fast ausschließlich über soziale Medien, also auch zu großen Teilen über Facebook. Andere Kanäle um auf Ihre technopolitische (politische Meinungsbildung, getrieben durch Technologie) Kampagne aufmerksam zu machen sind u.a. (Vgl. \cite{prasad2017}):
\begin{itemize}
  \item YouTube: z.B. durch Erklärvideos einer angeheuerten komödischen Gruppe
  \item Ihr eigener Blog: in welchem die Probleme in einer seriöseren Form geschildert werden
  \item Twitter:
    \begin{itemize}
      \item zur direkten Interaktion untereinander und mit den Anhängern
      \item der Konfrontation mit Befürwortern von Internet.org
      \item um bei Tech-Firmen Gutheißungen für die Bewegung zu erbitten
    \end{itemize}
\end{itemize}

% 2
\section{Lebenszyklus der Initative}

Nachdem nun die Ausgangslage geschildert wurde, wird in diesem Teil der Verlauf des Projektes von der Einführung bis zum Verbot nach verfolgt.```

\subsection{Übersicht}

\definecolor{BGColor}{rgb}{1.0,1.0,1.0}
\begin{timeline}{2015}{2017}{250}{145}
  \MonthAndYearEvent{2}{2015}{Formeller Start}
  \MonthAndYearEvent{4}{2015}{Partielle Einführung}
  \MonthAndYearEvent{7}{2015}{Umbenennung zu Free Basics}
  \MonthAndYearEvent{11}{2015}{Vollständige Einführung}
  \MonthAndYearEvent{2}{2016}{Verbot}
\end{timeline}

Nur ein Jahr hat es vom offizellen Start Anfang 2015, bis zum Verbot im Februar 2016 gedauert.

In der Zwischenzeit versuchte Facebook, nach initellen Startschwierigkeiten, die Initiative durch Rebranding zu retten, was aber nicht reichte um die kritische Öffentlichkeit zu beeinflußen, deren Meinung bis zu den Entscheidungträgern der zuständigen Behörde vordrigen konnte.
Welche sich dann mit einem indirekten Verbot auf der Seite der Protestierenden positionierte.

% \printbibliography

\end{document}
