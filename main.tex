\documentclass{article}
\usepackage[backend=biber]{biblatex}
\addbibresource{bibliography.bib}

\title{Wie entwickelte sich Facebooks Internet.org/Free Basics in Indien?}
\author{
  Fischer, Sarah Susanne\\
  \and
  Hein, Oliver\\
  \and
  Neidel, Jonathan\\
  \and
  Petersen, Tom Magnus\\
  \and
  Schmitz, Kai-Ibe\\
}

\begin{document}

\maketitle
% 1.2.2 in der Gliederung
\subsubsection{Internetverbreitung}

In Indien leben 2015 schon 350 Millionen Menschen\autocite{slideshareIndia} welche das Internet aktiv nutzen, wovon allein 159 Millionen per Mobilgerät unterwegs sind. Obwohl Indien damit schon den 2. Platz der aktiven Internetnutzer weltweit genießt, haben lediglich 27\% der indischen Bevölkerung Zugang zum Internet\autocite{InternetCountry}.
\medskip 
Ein Computer ist in Indien ein Luxusgut, welches durch die rasante Entwicklung von Mobilgeräten für die Jugend schnell an Bedeutung verlor, während Handys mit mobilem Internet allgegenwärtig wurden.
Durch die explosionsartige Verbreitung der neuen Kommunikationsmöglichkeiten entstanden in kürzester Zeit große Gruppen im Internet, welche sich über neue Technologien, Wissenswertes und Ihre Interessen austauschen.
Dieser rapide Aufbau von indivuellen Gruppen ist ein neues Phänomen, welches vorallem in `Entwicklungsländern' auftritt, in denen neue, und vergleichsweise preiswerte Technologien, den Zugang zum Internet ermöglichen. \textcite{empowermentThroughFacebook}
\medskip
Die durchschnittliche Bandbreite mobilen Internets war mit 2.8 Mb/s der durchschnittlichen Bandbreite eines Festanschlusses, von 2.3 Mb/s, überlegen \autocite{slideshareIndia}, wodurch das Interesse an einen Computer noch mehr abnahm.
Der Preis einer Prepaid Karte mit 1 GB Datenvolumen betrug derzeit ca. 3.55\% des durchschnittlichen Monatsgehalt\autocite{broadbandAgency}, und ist durch den hohen Preis nur für städtisch lebende Inder erschwinglich, da dort das Einkommen weitaus höher ist.
Obwohl gerademal 31,2\% der indischen Bevölkerung in städtischen Gebieten lebt, leben von den 350 Millionen aktiven Internetnutzern gerademal 61 Millionen auf dem Land\autocite{IndiaBevölkerung}. Somit benutzten 2015 gerademal 7.5 \%, auf dem Land lebende Inder, aktiv das Internet \autocite{slideshareIndia}.

Die fehlende weitreichende Abdeckung der Internetleitungen und der vergleichsweise hohe Datentarif sind schwerwiegende Probleme, welche Facebook inc. mit Internet.org lösen will.
% \printbibliography
\end{document}
