\documentclass{article}
\usepackage[backend=biber]{biblatex}
\addbibresource{bibliography.bib}

\title{Wie entwickelte sich Facebooks Internet.org/Free Basics in Indien?}
\author{
  Fischer, Sarah Susanne\\
  \and
  Hein, Oliver\\
  \and
  Neidel, Jonathan\\
  \and
  Petersen, Tom Magnus\\
  \and
  Schmitz, Kai-Ibe\\
}

\begin{document}

\maketitle
\subsubsection {Facebook}
Eine Studie von \textcite{empowermentThroughFacebook}, welche zwischen 2011 und 2012 in New Delhi durchgeführt wurde, zeigt, dass Facebook zu der Zeit eine große Rolle in der gesellschaftlichen Entwicklung spielte. 
\medskip
Da zu 2012 eine Regulierung von bis zu 200 SMS pro Tag veröffentlicht wurde\autocite{PressInformation}, stieg das Interesse von Facebook zu der Zeit stark an,
da Handybesitzer weiterhin unlimitiert mit ihren Freunden kommunizieren wollten.
Durch das Kennenlernen und Nutzen von Facebook entwickelte sich die indische Gesellschaft, welche bislang von der Globalisierung durch das Internet ausgeschlossen war, nun erstmals dahin, dass die Menschen sich eine transnationale Identität erstellen konnten.
\medskip

Die immernoch währende Kastengesellschaft und strenge Überwachung von Frauen in Indien führt dazu, dass 2015 von den 135 Millionen aktiven Facebook Nutzern gerademal 25 weiblich waren.

Das weitverbreitete Vorurteil, die Keuschheit von jungen Frauen sei gefährdet, wenn sie Zugang zu Handys hätten, macht es für diese geradezu unmöglich sich frei im Internet entfalten zu können, oder gar, ein Handy zu besitzen. Selbst von verheirateten Frauen, welche vom Mann die Erlaubnis ein Handy zu besitzen, wird nicht selten der Zugang zum Internet von den Schwiegeeltern überwacht.


Somit entwickelte sich Facebook zu einem Sozialem Netzwerk, welches durch die maskuline Dominanz für Frauen als gefährlich erachtet wird und es selbst den Frauen, welche Internetzugang haben, von den Familien verboten wird.


%viele brachten sich Englisch, die "Sprache des Internets", 
%selbst in ihrer Freizeit bei, um besser mit internationalen Kontakten
%kommunizieren zu können und auch ausländische News und Medien zu verstehen, ohne Google Übersetzer benutzen zu müssen.
\medskip
Seither blieb die Internetgeschwindigkeit ein Problem, mit welcher die Inder eine durchschnittliche Seitenladezeit von 3,9 Sekunden hatten\autocite{mashable}. Anfang 2015 veröffentlichte Facebook.inc die Facebook Messenger App, mit welcher sich nur Nachrichten und Bilder schicken ließen, und die Internetverbindung demnach nicht allzusehr belastet wird, aber eine Lösung schien das nicht zu sein.
\medskip
Um das Problem der langsamen Internetgeschwindigkeit anzugehen, überlegte sich Facebook das Konzept für Internet.org.

%%% marked with \"%" is going to be changed, deleted or replaced %%%

% \printbibliography
\end{document}
