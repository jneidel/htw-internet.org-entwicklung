\documentclass{article}
\usepackage[backend=biber]{biblatex}
\addbibresource{bibliography.bib}

\title{Wie entwickelte sich Facebooks Internet.org/Free Basics in Indien?}
\author{
  Fischer, Sarah Susanne\\
  \and
  Hein, Oliver\\
  \and
  Neidel, Jonathan\\
  \and
  Petersen, Tom Magnus\\
  \and
  Schmitz, Kai-Ibe\\
}

\begin{document}

\maketitle
\subsubsection Internetverbreitung

In Indien gibt es 2015 schon 350 millionen Menschen , welche das Internet aktiv nutzen, und davon 159 millionen per Handy \autocite{slideshareIndia}. 

Für die Jugend Indiens ist ein PC ein Luxusgut und kaum zu haben, aber durch den rapiden Anstieg der Technologie in Handys und Smartphones verlor sich das Interesse an einen PC sehr schnell und ein eigenes Handy zu besitzen wurde allgegenwärtig.\textcite{empowermentThroughFacebook}
% ^das werde ich noch umschreiben und mich auf dir Studie beziehen damit das nicht so aussieht ich habe mit das ausgedacht 
\medskip
Die durchschnittliche Bandbreite des mobilen Netzes war mit 2.8 Mb/s der durchschnittlichen festen Internetleitung überlegen, denn diese betrug lediglich 2.3 Mb/s\autocite{slideshareIndia}.
Da eine Prepaid Karte mit 1GB Datenvolum nur ca. 3.55\% des durchschnittlichen Monatsgehalt kosteten \autocite{broadbandAgency}, konnte sich so gut wie jeder Internet leisten.
Durch diese neue Möglichkeit miteinander zu kommunizieren und die Welt des Internets kennenzulernen bildeten sich große Gruppen für Wissensaustausch und Interessen. 
\medskip
Nicht zu übersehen sei dennoch, dass der Großteil der indischen Bevölkerung im ländlichen Gebiet lebt und gerademal 31,2 \% \autocite{IndiaBevölkerung} in städtischen Gebieten. 
Von den 350 millionen aktiven Internetnutzern leben also lediglich 61 millionen nicht in einer Stadt. Somit benutzten 2015 gerademal 7.5 \%, auf dem Land lebende Inder, aktiv das Internet \autocite{slideshareIndia}.
Grund dafür ist vorallem die fehlende Abdeckung mobilen Internets in ländlichen Gebieten, was dazu führt, dass vorallem Jugendliche vom Land in städtische Gebiete ziehen und somit die ländlichen Gebiete noch mehr dem Armut verfallen. 
%Überleitung zu Internet.org - > rural poor people + Internet.org = rural dich people?

% \printbibliography
\end{document}
