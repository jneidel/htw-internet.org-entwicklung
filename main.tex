\documentclass{article}
\usepackage[backend=biber]{biblatex}
\addbibresource{bibliography.bib}

\title{Wie entwickelte sich Facebooks Internet.org/Free Basics in Indien?}
\author{
  Fischer, Sarah Susanne\\
  \and
  Hein, Oliver\\
  \and
  Neidel, Jonathan\\
  \and
  Petersen, Tom Magnus\\
  \and
  Schmitz, Kai-Ibe\\
}

\begin{document}
\subsection{Beweggründe für Internet.org}
Facebook, ein soziales Netzwerk, wurde 2004 gegründet und ist seit dem, genauso wie das dahinterstehende Unternehmen Facebook, Inc., stetig gewachsen. 
Doch da die Anzahl von Anmeldungen neuer Internet-Kabelanschlüsse in der westlichen Welt seit 2008 gesunken sind
(Vgl. \cite{ICTslowingDown}), hat das Wachstum von Facebook dementsprechent auch nachgelassen.
Dies passiert obwohl ``at the beginning of 2016, only an estimated 3.2 billion people — 44 percent of the world’s population are online and connected to the digital economy." (\cite[7]{connectWorld})
Der Grund dafür: der Großteil der restlichen 56\% der Menschen lebt in Entwicklungsländern. Dort besteht meist entweder nicht die Möglichkeit das Internet zu nutzen (wegen zu hoher Kosten/keinem Internetzugang) oder man sieht den Nutzen des Internets nicht, da einem dieser nie vermittelt worden ist.

\medskip

Für Facebook, eine Firma welche durch Werbung und personenbezogene Daten Geld verdient, ist es von langfristigem Interesse, dass sich die Anzahl der Internetnutzer erhöht.
Es lässt sich ableiten, dass dafür das Internet in den Entwicklungsländern sich ausbreiten, und das Interesse der Menschen für die Anwendung geweckt werden muss.
Facebook ist nicht die einzige IT-Firma welche Interesse daran gezeigt hat.
Eine weitere ist die Dachgesellschaft von Google, sie hat 2011 mit ihren Tests für das Projekt Loon, welches das Ziel hat das Internet Netzwerk auf abgelegende Gegenden auszubreiten, angefangen.(Vgl. \cite{projectLoon})
Obwohl sie unterschiedliche Herngehensweisen und Schwerpunkte haben, sehen einige Journalsiten hierbei ein Rennen für die Internetverbreitung. 

\medskip
    
Im folgenden Text wird auf die Gründe die Facebook angegeben hat eingegangen, doch diese Gründe werden kritisch gesehen.   

\medskip

Mark Zuckerberg, Gründer und Vorstandsvorsitzender von Facebook Inc., hat 2013 das Dokument ``Is Connectivity a Human Right?" veröffentlicht.
In diesem, welches die Grundlage für Internet.orgs Konzept ist, erläutert er, dass es für Facebook wahrscheinlich nicht gewinnbringend sei, das Internet in den Entwicklungsländern auszubreiten. Trotzdem arbeitet Facebook daran dies zu machen, da er daran glaubt, dass jeder Mensch es verdient hat verknüpft zu sein.(Vgl. \cite[1]{HumanRight})
Angebliche positive Auswirkungen durch eine Ausbreitung des Internets, welches eine Bildungs- und Verbindungsmöglichkeit darstellt, in Ländern der dritten Welt, seien die Entstehung neuer Arbeitsplätze, der Rückgang der Armut und die positive Entwicklung der Wirtschaft. Diese Auswirkungen werden in einem von Facebook in Auftrag gegebenen Report verdeutlicht:
                    
``[...] we believe that achieving universal Internet penetration could expand world output by \$6.7 trillion.  
In addition, global inclusion could bring 7 percent of the world’s population above absolute poverty levels — in effect providing 500 
million of the world’s poorest people new opportunities to grow and engage with the world"
- \cite[11]{connectWorld}

\medskip

Mark Zuckerberg ist weiterhin der Meinung, dass die Grundlage für eine zukünftige wissenbassierte Wirtschaft, die globale Vernetzung der Menschheit sei \parencite{HumanRight}.
Die Menschen, die man online bringen würde, würden viele neue Ideen, Dienstleistungen und Produkte bereitstellen, durch welche die ganze Welt profitieren würde.
Facebooks Ziel ist also, nach eigener Angabe, das beste für die Welt zu erzielen.
% \printbibliography
\end{document}
