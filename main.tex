\documentclass{article}

\usepackage[
backend=biber,
style=authoryear,
citestyle=authoryear,
autocite=footnote
]{biblatex}
\addbibresource{bibliography.bib}

\title{Wie entwickelte sich Facebooks Internet.org/Free Basics in Indien?}
\author{
  Fischer, Sarah Susanne\\
  \and
  Hein, Oliver\\
  \and
  Neidel, Jonathan\\
  \and
  Petersen, Tom Magnus\\
  \and
  Schmitz, Kai-Ibe\\
}

\begin{document}

\maketitle

% 1.2.4 in der Gliederung
\subsubsection{Aktivisten}

% go into their arguments?

Die Seite welche gegen Facebook argumentiert sind die Net Neutrality Aktivisten, vereint unter dem Banner der 'Save the Internet' Kampagne.
Zusammengesetzt aus Technologieinteressierten, primär den Mitarbeitern von Tech-Startups, die sich selbst als ``geeks and enthusiasts from various fields: technology, law, journalism, design, policy" \parencite{sti2015} beschreiben.

\medskip

Die Aktivisten können als eine 'recursive public' verstanden werden. Welche von \textcite{twoBits} wie folgt definiert wird:

\medskip

``A recursive public is a public that is constituted by a shared concern for maintaining the means of association through which they come together as a public.
Geeks find affinity with one another because they share an abiding moral imagination of the technical infrastructure, the Internet, that has allowed them to develop and maintain this affinity in the first place." - \cite[28]{twoBits}.

\medskip

Dies deckt sich mit dem Handeln der Aktivisten, welche sich über online Plattformen kennengelernt und unter dem gemeinsamen Ziel - das Internet wie sie es kennen (und lieben) zu verteidigen - vereinigt haben.

\medskip

Obwohl die Mitglieder von 'Save the Internet' nur einen kleinen Teil der Bevölkerung repräsentieren, nämlich den der high-tech Arbeiter deren Anteil bei 2\% liegt (in einem Land in welchem 70\% der Menschen landwirtschaftlichen Tätigkeiten nachgehen), haben sie doch eine besondere Position inne. Denn sie spielen eine zentrale Rolle in der Vorstellung des neuen Indiens, welches in der globales Wirtschaft mitmischen kann (Vgl. \cite{thomas2012}).
Sie sind das Symbol der aufsteigenden Mittelschicht, welche durch Unternehmerschaft und harte Arbeit die indische Wirtschaft (7,7\% des 2,6 Billionen BIP 2017 \autocite{statistaIndiaGDP}\autocite{imfIndiaGDP}) und Indiens Position in der Welt voran treiben (8\% der Startups im Silicon Valley der 90er Jahre wurden von Indern angeführt \parencite{upadhya2004}).

\medskip

In Ihrer öffentlicher Präsentation ist die Initiative angelehnt an die 'Net Neutrality' Bewegung der USA: von der Namensgebung und dem Logo über die verwendeten Hashtags in den sozialen Medien bis zu dem Vokabular mit welchem sie arbeiten (Vgl. \cite{prasad2017}). Diese bewusst gewählte Assoziation mit dem erfolgreichen amerikanischen Modell und auch die direkten Verweise auf westliche Popkultur soll die Kampagne für Ihre Zielgruppe ansprechender machen.

Die Zielgruppe setzt sich nicht aus den Menschen denen Free Basics helfen möchte, sondern aus dem indische Mittelstand. Denn die Sprache mit der die Menschen angesprochen werden ist 'Hinglish', ein Mix aus Hindi und Englisch, für deren Verständnis sowohl Englischkenntnisse - welche Zugang zu neuen, qualifizierten Jobs ermöglichen, sowie eine Nähe zum Westen definieren die neue Mittelschicht Indiens (Vgl. \cite{fernandes2006}) - als auch ein hohen Sprachniveau in Hindi voraussetzt.

\medskip

Ironischerweise, obwohl die Aktivisten gegen Facebook wettern, verbreiten sie Ihre Botschaft fast ausschließlich über soziale Medien, also auch zu großen Teilen über Facebook. Andere Kanäle um auf Ihre technopolitische (politische Meinungsbildung, getrieben durch Technologie) Kampagne aufmerksam zu machen sind u.a. (Vgl. \cite{prasad2017}):
\begin{itemize}
  \item YouTube: z.B. durch Erklärvideos einer angeheuerten komödischen Gruppe
  \item Ihr eigener Blog: in welchem die Probleme in einer seriöseren Form geschildert werden
  \item Twitter:
    \begin{itemize}
      \item zur direkten Interaktion untereinander und mit den Anhängern
      \item der Konfrontation mit Befürwortern von Internet.org
      \item um bei Tech-Firmen Gutheißungen für die Bewegung zu erbitten
    \end{itemize}
\end{itemize}

% \printbibliography

\end{document}
