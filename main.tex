\documentclass{article}
\usepackage[
backend=biber,
style=authoryear,
citestyle=authoryear,
autocite=footnote
]{biblatex}
\usepackage{packages/timeline}
\addbibresource{bibliography.bib}

\title{Wie entwickelte sich Facebooks Internet.org/Free Basics in Indien?}
\author{
  Fischer, Sarah Susanne\\
  \and
  Hein, Oliver\\
  \and
  Neidel, Jonathan\\
  \and
  Petersen, Tom Magnus\\
  \and
  Schmitz, Kai-Ibe\\
}

\begin{document}

\maketitle

% 1
\section{Ausgangsbedingungen}

Zunächst wird ein Überblick über die Ausgangsbedingungen in Indien geschaffen. Von der Internetverbreitung, Ihrer Beziehung zu Facebook, vorherigen Diskussionen zur Netzneutralität sowie einer Beschreibung der Aktivisten, welche sich gegen Facebooks Initiative wendeten.

Als Erstes wird nun aber das Konzept für Internet.org/Free Basics vorgestellt, gefolgt von einer Beleuchtung von Facebooks Beweggründe für die Gründung der Initiative.

% 1.1
\subsection{Internet.org/Free Basics}

% 1.1.2
\subsubsection{Konzept}
        
Mark Zuckerberg nennt drei Punkte, an denen man arbeiten muss, damit sich das Internet in den Entwicklungsländern ausbreiten kann
\parencite{HumanRight}:
\begin{itemize}
\item ``making internet access affordable by making it more efficient to deliver data."

\item ``using less data by improving the efficiency of the apps and experiences we use."

\item ``helping businesses drive internet access by developing a new model to get people online."  
\end{itemize}

Die ersten zwei Punkte, die er erwähnt, zielen darauf ab das Internet kostengünstiger/kostenlos anzubieten. 
Dies ist notwendig damit Menschen, die nur ein geringes Einkommen haben, es auch nutzen können.

\medskip

Zunächst müsste man den Datenverkehr effizienter gestalten, zum Beispiel durch leistungsfähigere Hardware.
Dadurch würden sich die Kosten für die Infrastruktur verringern und die Unternehmen, die diese bereitstellen, würden mehr Profit    
machen. Dies würde zu einer Senkung der Internetkosten für den Einzelnen führen.
Zusätzlich könnten die Unternehmen den überschüssigen Profit zum weiteren Ausbau des Netzwerkes benutzen\\

Facebook selber möchte auch an dem Ausbau der Internet Infrastruktur arbeiten ( mithilfe von Drohnen und Satelliten - die obtische    
Freiraumkommunikation benutzen), aber das Primärziel ist es Menschen die bereits in Reichweite von Internetzugängen leben, zu
verbinden.\\
Also wollen sie sich darauf konzentrieren die Kosten für die Telekommunikationsfirmen zu senken.
Sie sind bereits Mitwirkende an dem Open Compute Project, welches ein Projekt ist um Leistungsfähige und kostengünstige Hardware 
Design mit unterschiedlichen Firmen zu teilen. \\
Facebook arbeitet außerdem daran web-caching(Zwischenspeicherung der benötigten Daten in den Datenzentren) den grundlegenden 
Internetdienstleistungen anzubieten, dadurch wird Zeit gespart und der Datenverkehr verringert sich.\\

Facebook nimmt an, dass mit einem organisierten Einsatz es möglich sei, die Leistungsfähigkeit von dem Zustellen von Daten in den   
nächsten 5 bis 10 Jahren um das 100 fache zu erhöhen  (vgl. \cite{HumanRight})\\

\medskip

Der zweite Punkt auf den Zuckerberg eingeht konzentriert sich auf die Leistungsfähigkeit der angebotenen Dienstleistungen und wie man    diese verbessern kann. Es ist logisch: wenn man weniger Daten für die Internetnutzung braucht ist sie kostengünstiger.\\

Dienstleistungen kann man auf verschiedenen Weisen effizienter gestalten\\

Mark Zuckerberg sagt ``At Facebook, we’re investing heavily in opportunities to reduce our overall data use and help  other apps     
reduce their data use as well. Some of the areas we’re focused on are caching, data compression and simple efficiency optimizations.
(\cite{HumanRight})\\

Die Daten Komprimierung kann man entweder durch die direkte Kompression von großen Apps realisieren oder man könnte neue Dienstleistungen entwickeln, durch welche man seine Daten schicken und Komprimieren lassen kann.\\

\medskip

Mit dem dritten und letzten Punkt hat Zuckerberg eine spezifische Sache umschrieben.
Das ``new model" von dem er spricht kann man in einem Satz zusammenfassen:
Die Menschen, in den Entwicklungsländern sollen erst mal nur den kostenlosen Zugang zu Dienstleistungen wie Facebook haben.\\
Für diese Herangehensweise wurden zwei Begründungen gegeben:\\

Trotz der Senkung der Internet Kosten kann nicht jegliche Dienstleistung kostenlos bereitgestellt werden, da die dahinterstehende Anbieter Profit erzielen müssen. 
Aber durch das kostenlose Bereitstellen grundlegender Dienstleistungen könnte man gewährleisten, dass die meisten Menschen Zugang zum Internet haben und die Industrie trotzdem gleichzeitig den größten wirtschaftlichen Profit erzielt (wodurch diese wiederum mehr Zugang zum Internet bieten kann).\\

Obwohl Zuckerberg sagt, dass er keine spezifische Gruppe von grundlegenden Internet Dienstleistungen vorgibt, beschreibt er welche Anwendungen er als grundlegend ansieht.(\cite{HumanRight})\\
Seiner Meinung nach sind es die Dienstleistungen, die nicht Daten intensiv sind (hauptsächlich Text basierende Dienstleistungen und einfach gehaltene Apps). 
Außerdem sollte man durch die Anwendungen anderen Inhalt entdecken und sinnvoll mehr Daten benutzen können.
Dienstleistungen die unter diesen Kriterien fallen sind zum Beispiel Messengers, soziale Netzwerke, Suchmaschinen und Wikipedia.\\

Also ist Facebook nach Zuckerbergs Meinung eine grundlegende Internet Dienstleistung die vorrangig zu anderen Anwendungen kostenlos zur Verfügung gestellt werden sollte.
Ein weiterer Grund den Zuckerberg für den partiellen kostenlossen Zugang erwähnt hat ist:\\
"many people who have never experienced the internet don’t know what a data plan is or why they’d want one. 
However, most people have heard of services like Facebook and messaging and they want access to them. If we can provide people 
with access to these services, then they’ll discover other content they want and begin to use 
and understand the broader internet." \parencite{HumanRight}

\medskip

Facebooks Plan ist es, unter dem Projekt Internet.org, diese Punkte Industrie übergreifend zu bearbeiten.
Auf der offizielen Facebook Seite für Internet.org wurde das Projekt wie folgt vorgestellt:\\
``Internet.org is a global partnership between technology leaders, nonprofits, local communities and experts who are working together to bring the internet to the two thirds of the world's population that doesn't have it. Sharing tools, resources and best practices, Internet.org partners will explore solutions in three major opportunity areas: affordability, efficiency and business models.\#ConnectTheWorld" (\cite{InternetOrg})\\

An vielen von diesen Ideen hat Facebook bereits vor der Gründung von Internet.org, welche 2013 mit 6 anderen IT-Firmen (Opera Software, Nokia, Samsung, Ericsson, MediaTek und Qualcomm) stattgefunden hat, gearbeitet.
Facebook hatte die Projekte Facebook Zero (Bereitstellung einer kostenlosen Internetverbindung solange man auf Facebooks Webseiten ist) und ``Facebook für jedes Handy" (Version von Facebook für feature Phones) 2010 und 2011 eingeführt.
Bei beiden dieser Projekte hat Facebook bereits oben genannte Mittel verwendet um den Datenverkehr effizienter zu gestalten.
Diese Technologie will Facebook nun allgemein anwendbar machen, damit andere Dienstleistungen diese auch benutzen können.

% 1.2.3
\subsubsection{Netzneutralität}

Netzneutralität, oder `Net Neutrality' im Englischen, beschreibt ein Prinzip, nach welchem ein Internetanbieter allen Datenverkehr gleich behandelt, dass heißt ohne bestimmte Internetangebote langsamer oder teurer zu machen als andere \autocite{netzneutralität}.

\medskip

Indiens Vergangenheit ist im Bezug auf Netzneutralität noch sehr jung. So gab es vor Internet.org keine bestehende Verordnungen, und nur einige Vorfälle, welche eine Diskussion des Themas veranlassten.
Beide relevanten Debatten, ausgelöst vom lokalen Mobilfunkprovider Airtel, zielten darauf entgangene Gewinne des Konzerns ausgleichen zu lassen.
So wurde 2012 die Möglichkeit zur Besteuerung von Youtube, Google, Facebook, und Co. vorgeschlagen um für deren großen Teil am Datenverkehr zu kompensieren.
Auch 2014 ging es um eine ähnliche Problematik: so sollten VoIP (Voice over IP, Internettelefonie) Daten teurer sein als sonstige mobile Daten.
Beide Vorschläge führten zu keinem Ergebnis und wurden zum Teil stark von der Öffentlichkeit protestiert (Vgl. \textcite[253]{everydayLife} und \textcite{airtelVoip}).

% 1.2.4
\subsubsection{Aktivisten}

Die Seite, welche gegen Facebook argumentiert, sind die `Net Neutrality' Aktivisten, vereint unter dem Banner der `Save the Internet' Kampagne.
Zusammengesetzt aus Technologieinteressierten, primär den Mitarbeitern von Tech-Startups, die sich selbst als ``geeks and enthusiasts from various fields: technology, law, journalism, design, policy" \parencite{sti2015} beschreiben.

\medskip

Die Aktivisten können als eine `recursive public' verstanden werden. Welche von \textcite{twoBits} wie folgt definiert wird:

\medskip

``A recursive public is a public that is constituted by a shared concern for maintaining the means of association through which they come together as a public.
Geeks find affinity with one another because they share an abiding moral imagination of the technical infrastructure, the Internet, that has allowed them to develop and maintain this affinity in the first place." - \cite[28]{twoBits}.

\medskip

Dies deckt sich mit dem Handeln der Aktivisten, welche sich über online Plattformen kennengelernt und unter dem gemeinsamen Ziel - das Internet wie sie es kennen (und lieben) zu verteidigen - vereinigt haben.

\medskip

Obwohl die Mitglieder von `Save the Internet' nur einen kleinen Teil der Bevölkerung repräsentieren, nämlich den der high-tech Arbeiter deren Anteil bei 2\% liegt (in einem Land in welchem 70\% der Menschen landwirtschaftlichen Tätigkeiten nachgehen), haben sie doch eine besondere Position inne. Denn sie spielen eine zentrale Rolle in der Vorstellung des neuen Indiens, welches in der globales Wirtschaft mitmischen kann (Vgl. \cite{thomas2012}).
Sie sind das Symbol der aufsteigenden Mittelschicht, welche durch Unternehmerschaft und harte Arbeit die indische Wirtschaft (7,7\% des 2,6 Billionen BIP 2017 \autocite{statistaIndiaGDP}\autocite{imfIndiaGDP}) und Indiens Position in der Welt voran treiben (8\% der Startups im Silicon Valley der 90er Jahre wurden von Indern angeführt \parencite{upadhya2004}).

\medskip

In Ihrer öffentlichen Präsentation ist die Initiative angelehnt an die `Net Neutrality' Bewegung der USA: von der Namensgebung und dem Logo über die verwendeten Hashtags in den sozialen Medien bis zu dem Vokabular mit welchem sie arbeiten (Vgl. \cite{prasad2017}).
Diese bewusst gewählten Assoziation mit dem erfolgreichen amerikanischen Modell, und auch die direkten Verweise auf westliche Popkultur, sollen die Kampagne für Ihre Zielgruppe ansprechender machen.

Die Zielgruppe setzt sich nicht aus den Menschen zusammen welchen Free Basics helfen möchte, sondern aus dem indische Mittelstand. Denn die Sprache mit der die Menschen angesprochen werden ist `Hinglish', ein Mix aus Hindi und Englisch, für deren Verständnis sowohl Englischkenntnisse - welche den Zugang zu neuen, qualifizierten Jobs ermöglichen und, ebenso wie eine Nähe zum Westen, die neue Mittelschicht Indiens prägen (Vgl. \cite{fernandes2006}) - als auch ein hohen Sprachniveau in Hindi voraussetzt.

\medskip

Ironischerweise, obwohl die Aktivisten gegen Facebook wettern, verbreiten sie Ihre Botschaft fast ausschließlich über soziale Medien, also auch zu großen Teilen über Facebook. Andere Kanäle um auf Ihre technopolitische (politische Meinungsbildung, getrieben durch Technologie) Kampagne aufmerksam zu machen sind u.a. (Vgl. \cite{prasad2017}):
\begin{itemize}
  \item YouTube: z.B. durch Erklärvideos einer angeheuerten komödischen Gruppe
  \item Ihr eigener Blog: in welchem die Probleme in einer seriöseren Form geschildert werden
  \item Twitter:
    \begin{itemize}
      \item zur direkten Interaktion untereinander und mit den Anhängern
      \item der Konfrontation mit Befürwortern von Internet.org
      \item um bei Tech-Firmen Gutheißungen für die Bewegung zu erbitten
    \end{itemize}
\end{itemize}

% 2
\section{Lebenszyklus der Initative}

Nachdem nun die Ausgangslage geschildert wurde, wird in diesem Teil der Verlauf des Projektes von der Einführung bis zum Verbot nach verfolgt.```

\subsection{Übersicht}

\definecolor{BGColor}{rgb}{1.0,1.0,1.0}
\begin{timeline}{2015}{2017}{250}{145}
  \MonthAndYearEvent{2}{2015}{Formeller Start}
  \MonthAndYearEvent{4}{2015}{Partielle Einführung}
  \MonthAndYearEvent{7}{2015}{Umbenennung zu Free Basics}
  \MonthAndYearEvent{11}{2015}{Vollständige Einführung}
  \MonthAndYearEvent{2}{2016}{Verbot}
\end{timeline}

Nur ein Jahr hat es vom offizellen Start Anfang 2015, bis zum Verbot im Februar 2016 gedauert.

In der Zwischenzeit versuchte Facebook, nach initellen Startschwierigkeiten, die Initiative durch Rebranding zu retten, was aber nicht reichte um die kritische Öffentlichkeit zu beeinflußen, deren Meinung bis zu den Entscheidungträgern der zuständigen Behörde vordrigen konnte.
Welche sich dann mit einem indirekten Verbot auf der Seite der Protestierenden positionierte.

% \printbibliography

\end{document}
