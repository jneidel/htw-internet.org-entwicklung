\documentclass{article}
\usepackage[backend=biber]{biblatex}
\addbibresource{bibliography.bib}

\title{Wie entwickelte sich Facebooks Internet.org/Free Basics in Indien?}
\author{
  Fischer, Sarah Susanne\\
  \and
  Hein, Oliver\\
  \and
  Neidel, Jonathan\\
  \and
  Petersen, Tom Magnus\\
  \and
  Schmitz, Kai-Ibe\\
}

\begin{document}

\maketitle
% 2.3 in der Gliederung
\subsection{Start in ausgewählten Regionen}

Im Anschluss an den formalen Start begann Facebook mit dem Aufsetzen der Netzinfrastruktur in den Regionen Tamil Nadu, Maharashtra, Andhra Pradesh, Gujarat, Kerala und Telangana.
Gleichzeitig fingen erneute Debatten um `Net-Neutrality' an Boden zu gewinnen.
Facebooks Dienst sei nicht konform mit dem Bestreben eines freien Internets, trotz der Bewerbung des Produkts als ``kostenloses Facebook" \parencite[3]{prasad2017}, später als ein ``step towards digital equality" (Ausführung in \ref{rebranding}).

\medskip

In der ländlichen Bevölkerung gab es auch geteilte Meinungen hinsichtlich des Zugangs zum Internet, zumindest bei jenen, die sich selbst informiert haben oder informiert werden (von z.B. Nachrichtendiensten wie Digit.in) \parencite{digitYT}.
Hier kommt es jedoch zumeist zur Ignoranz des großen Ganzen, der Dienst wird als positiv entgegengenommen und genutzt, Aktionen wie die wortwörtlich vorgeschriebene Unterschreibung zugunsten von Facebook in der Debatte um Netzneutralität \textcite{ndtvYT} werden ohne weiteres Hinterfragen und potenziell aufgrund von Unwissen unterstützt.
Einzig ein initielles Interesse an einem kostenlosen Dienst im allgemeinen Sinne ist ein gemeines Ziel.

\medskip

Anzumerken bei alledem ist, dass der Großteil der Bevölkerungsgruppe, die in Facebooks Werbekampanien als Zielgruppe des Dienstes beschrieben werden, entweder keinen, oder nur einen gerinfügigen Nutzen genießt.
Jene, die sich den Besitz eines digitalen Endgeräts (id est Smartphones, etc.) leisten können, sind aufgrund der geringen Preise auch in der Lage, eine stabilere und schnellere Internetverbindung als Facebooks Dienst sie anbietet, zu erwerben.
Der Rest, also Personen, die (zu Teilen weit) unterhalb der Armutsgrenze leben, haben weder Wege, FreeBasics zu nutzen, noch einen Grund dafür \parencite[257]{everydayLife}.

\medskip

So ist die Nutzung des Dienstes selbst fern vom beinahe utopischen Beispiel von Facebooks Seite des Landwirts Ganesh, welcher mithilfe von Free Basics seinen Ertrag verdoppelte, indem er Informationen über das Wetter und aktuelle Warenpreise nutzt, ein Beispiel für die Angabe, dass für alle 10 Teilnehmer an Internet.org eine Person aus der Armut gehoben wird \parencite[4]{prasad2017}.
In der Realität besteht die Nutzerbasis wie besagt aus männlichen Jugendlichen und jungen Erwachsenen.
Hier wurde Facebook zum Teil ein integrierter Bestandteil des Lebens, der sich fast virusartig durch persönliche Weiterempfehlung verbreitete.
Dies geht über zu spezialisierten Kursen von neuen oder schon vorher "Tech-Affinen", die Teilnehmern Grundlegendes von der Inbetriebnahme von Smartphones bis zum Kommentieren bei Facebook beibringen \parencite{empowermentThroughFacebook}.

\medskip

Dies allein ist natürlich keine schlechte Sache; besagte Menschen gehören zu einer Generation von gelehrten, jedoch aufgrund der hohen Mensch- zu Arbeitsplatzratio zumeist arbeitslosen, zwanzig- bis dreißigjährigen Indern, die sich von der Welt abgehängt fühlen.
Eine Lehrfunktion bietet eben solchen neben Arbeit auch einen Daseinszweck.

\medskip

Besagte Kurse brachten einen weiteren positiven Effekt mit sich, die Verbreitung des Englischen durch Selbstbeibringung in Gebieten, in denen vorher weder Englisch noch Hindi gesprochen wurde. \cite{empowermentThroughFacebook} Die Tatsache, dass der Großteil der Analphabeten vor Allem unter der genannten Gruppe, welche unter der Armutsgrenze lebt, zu finden sind, bleibt hierbei leider bestehen.

\medskip

Ferner stehen positiven Aspekten weitere bereits etablierte soziale Normen im Wege; Informationsverbreitung geschieht im ländlichen Indien immer noch zu großen Teilen über Mundpropaganda \cite[259]{everydayLife}, Mädchen wird Zugriff auf Medien zugunsten des (erstgeborenen) Jungen verwährt \cite{empowermentThroughFacebook}, soziale Gruppierungen werden vom Realen ins Virtuelle übernommen.

\medskip

All dies hätte sich nicht durch eine ausschließlich kostenpflichtige Variante des Internets geändert.
Allein die Zahlen der nationalen Internetverbreitungsrate (2016 25\% $\to$ 2019 50\%) sprechen für eine graduelle Veränderung gen Akzeptanz bezüglich des Internets.
%I think this should be left out - or changed significantly - need opinions

\parencite[Vgl.]{prasad2017}
\parencite{digitYT}
\parencite{ndtvYT}

% \printbibliography
\end{document}

