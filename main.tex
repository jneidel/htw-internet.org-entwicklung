\documentclass{article}
\usepackage[backend=biber]{biblatex}
\addbibresource{bibliography.bib}

\title{Wie entwickelte sich Facebooks Internet.org/Free Basics in Indien?}
\author{
  Fischer, Sarah Susanne\\
  \and
  Hein, Oliver\\
  \and
  Neidel, Jonathan\\
  \and
  Petersen, Tom Magnus\\
  \and
  Schmitz, Kai-Ibe\\
}

\begin{document}

\maketitle

% 1.2.3 in Gliederung
\section{Netzneutralität}

Netzneutralität, oder `Net Neutrality' im Englischen, beschreibt ein Prinzip, nach welchem ein Internetanbieter allen Datenverkehr gleich behandelt, dass heißt ohne bestimmte Internetangebote langsamer oder teurer zu machen als andere \autocite{netzneutralität}.

% \printbibliography
\end{document}
