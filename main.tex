\documentclass{article}
\usepackage[backend=biber]{biblatex}
\addbibresource{bibliography.bib}

\title{Wie entwickelte sich Facebooks Internet.org/Free Basics in Indien?}
\author{
  Fischer, Sarah Susanne\\
  \and
  Hein, Oliver\\
  \and
  Neidel, Jonathan\\
  \and
  Petersen, Tom Magnus\\
  \and
  Schmitz, Kai-Ibe\\
}

\begin{document}

\maketitle

% 1.2.3 in Gliederung
\section{Netzneutralität}

Netzneutralität, oder `Net Neutrality' im Englischen, beschreibt ein Prinzip, nach welchem ein Internetanbieter allen Datenverkehr gleich behandelt, dass heißt ohne bestimmte Internetangebote langsamer oder teurer zu machen als andere \autocite{netzneutralität}.

\medskip

Indiens Vergangenheit, im Bezug auf Netzneutralität ist, anders als die der USA, noch sehr jung. So gab es vor Internet.org weder bestehende Verordnungen und auch nur einige Vorfälle welche eine Diskussion des Themas veranlassten.
Beide relevanten Debatten, ausgelöst vom lokalen Mobilfunkprovider Airtel, zielten darauf verlorene Gewinne des Konzerns ausgleichen zu lassen.
So wurde 2012 die Möglichkeit zur Besteuerung von Youtube, Google, Facebook, und Co. vorgeschlagen um für deren großen Teil am Datenverkehr zu kompensieren.
Auch 2014 ging es um eine ähnliche Problematik: so sollten VoIP (Voice over IP, Internettelefonie) Daten teurer sein als sonstige mobile Daten.
Beide Vorschläge führten zu keinem Ergebnis und wurden Teil stark von der Öffentlichkeit protestiert (Vgl. \textcite[253]{everydayLife} und \textcite{airtelVoip}).

% \printbibliography
\end{document}
