\documentclass{article}
\usepackage[backend=biber]{biblatex}
\addbibresource{bibliography.bib}

\title{Wie entwickelte sich Facebooks Internet.org/Free Basics in Indien?}
\author{
  Fischer, Sarah Susanne\\
  \and
  Hein, Oliver\\
  \and
  Neidel, Jonathan\\
  \and
  Petersen, Tom Magnus\\
  \and
  Schmitz, Kai-Ibe\\
}

\begin{document}

\maketitle
\section {Einleitung}
Die folgende Arbeit beschäftigt sich mit der Entwicklung von `Internet.org" , welches Mark Zuckerberg als großes Projekt in Indien durchführen wollte, aber daran kläglich scheiterte.
Die explosionsartige Verbreitung von Facebook im städtischem Indien sorgte für Aufmerksamtkeit in der Welt. Da aber das indische Internet auf Entwicklungsländischem Niveau stand, plant Mark Zuckerberg einen großen Eingriff in dieses Problem und stellt der indischen Bevölkerung Internet.org vor. Dieses verspricht den kostenlosen Zugriff auf das Internet, welches vorallem der unteren Gesellschaft helfen soll sich über effizientere Arbeitsweisen informieren zu können. Auch setzte es sich als Ziel, der Aufklärung zu dienen, und somit die bislang herrschende Kastengesellschaft aufzulockern.
Wie diese Idee bei der Bevölkerung ankommt, inwiefern Facebook.inc dieses Konzept umsetzt und anschließend nach kurzer Zeit schon von der Regierung gebannt wird, wird in dieser Arbeit gezeigt. 
%hört sich so an als würde Facebook.inc gebannt werden oder ? 
